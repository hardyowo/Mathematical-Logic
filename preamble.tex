
% Encoding Settings
\usepackage[T1]{fontenc}
\usepackage[utf8]{inputenc}



% Page Format Settings
% Enable multicolumn format in the page
\usepackage{multicol}
\usepackage{color}  % For color used column separator
\setlength{\columnsep}{0.2cm}  % Set Column separation
\setlength{\columnseprule}{1pt}
\def\columnseprulecolor{\color{blue}}  % Add a vertical line between two columns.

% Command template:
% \begin{multicols}{4}  % Four column format

% \end{multicols}

% Define a new font called Tinyb. This font can maintain its shape even in very small fontsize:
% \usepackage{lmodern}
% \rmfamily
% \DeclareFontShape{T1}{lmr}{bx}{sc}{<-> cmr10}{}% USE BOLD SCSHAPE NOT OTHERWISE DEFINED
% %%% MATH FONT FIX
% \DeclareFontFamily{OML}{zlmm}{}
% \DeclareFontShape{OML}{zlmm}{m}{it}{<-> lmmi10}{}
% \DeclareFontShape{OML}{zlmm}{b}{it}{<->ssub * zlmm/m/it}{}
% \DeclareFontShape{OML}{zlmm}{bx}{it}{<->ssub * zlmm/m/it}{}
% \DeclareMathVersion{Tinyb}
% \SetSymbolFont{operators}{Tinyb}{T1}{lmr}{bx}{sc}
% \SetSymbolFont{letters}{Tinyb}{OML}{zlmm}{m}{it}
% \newenvironment{tinyb}{\bgroup\tiny\bfseries\scshape\mathversion{Tinyb}}{\egroup}



% Paragraph Settings
\usepackage{setspace}
\onehalfspacing
\usepackage{parskip}
\setlength{\parindent}{0pt}



% Reference Settings
\usepackage[
    backend=biber,
    style=ieee,
    maxbibnames=3, % Maximum number of authors to be displayed in the bibliography
    citestyle=numeric-comp, % Put multiple citations in one bracket
]{biblatex}
\addbibresource{References.bib}

% Elegantly break long doi field
% \setcounter{biburllcpenalty}{100}
% \setcounter{biburlucpenalty}{100}
% \setcounter{biburlnumpenalty}{100}



% Figure Settings
\usepackage{graphicx} % Required for inserting images
\renewcommand{\figurename}{\textbf{Fig.}}

\usepackage{float}

% Enable two figures in one line:
% \usepackage{subfig}
% \begin{figure}[htbp!]
%   \centering
%   \subfloat[Before re-initialization]{\includegraphics[width=0.5\textwidth]{Figures/Before re-initialization.pdf}\label{fig:f1}}
%   \hfill
%   \subfloat[After re-initialization]{\includegraphics[width=0.5\textwidth]{Figures/After re-initialization.pdf}\label{fig:f2}}
%   \centering
%   \caption{Swarm distribution before and after reinitializing transitional particles on a one-dimensional problem}
%   \label{fig:before_and_after_reinitialization}
% \end{figure}



% Array Settings
\usepackage{array}
\newcolumntype{R}{>{$}r<{$}} % math-mode version of "r" column type
\newcolumntype{C}{>{$}c<{$}} % math-mode version of "c" column type
\newcolumntype{L}{>{$}l<{$}} % math-mode version of "l" column type

% defines a new type of column called Y based on a X column (this column type is defined by the tabularx package and it is basically a p{ <width>} column, where <width> is calculated by the package) but typesets the content using \small font size and with ragged-right text.
% \newcolumntype{Y}{>{\small\raggedright\arraybackslash}X}

% Modify the space on the bottom and top of each cell:
\usepackage{cellspace}
% \addtolength{\cellspacetoplimit}{5pt}
% \addtolength{\cellspacebottomlimit}{5pt}

\usepackage{multirow}
\usepackage{tabularx,booktabs}
\usepackage{longtable}
\usepackage{relsize}

% \renewcommand{\thetable}{S\arabic{table}}  % Rename the table names to Table Sx.

% Equally spread columns to fulfill the whole table.
% \begin{longtable}[c]{@{\extracolsep{\fill}}Lllllllll}

% Define a horizontal line that only appears in specific columns:
% \usepackage{hhline}
% \hhline{~----~~}  % Use as \hline, but the column with ~ will not have a horizontal line.



% Hyper-reference Settings
\usepackage{hyperref}
\hypersetup{
    colorlinks=true,
    linkcolor=cyan,
    citecolor=cyan,
    urlcolor=cyan,
}
% \usepackage[all]{hypcap}
% \makeatletter
% \AtBeginDocument{\def\@citecolor{cyan}}  % Define citing
% \AtBeginDocument{\def\@urlcolor{cyan}}
% \AtBeginDocument{\def\@linkcolor{cyan}}
% \makeatother

% Make the brackets of equation citation blue:
% \hyperref[eq:clpso_velocity]{(\ref*{eq:clpso_velocity})}



% Mathematical Settings
\usepackage{mathtools}
\usepackage{amssymb,mathrsfs}  % Typical maths resource packages
\usepackage{amsthm}
\usepackage{amsmath}
\usepackage{nccmath}  % To narrow parskip between two equations. \useshortskip

% Narrow paragraph skip between equation and its previous paragraph
% Command template:
% \useshortskip

% Enable the usage of probability P and E
% \mathds{P}
% \mathbb{E}



% Equation Settings
% \counterwithout{equation}{chapter}

% To use the large bracket on one side of equations:
% \begin{equation}
%     \label{eq:sum}
%     Sum =
%     \begin{cases}
%        Y_1 + Y_2 + Y_3, & \theta = 3 \\
%        Y_1 + Y_2 + \cdots + Y_8, & \theta = 8
%     \end{cases}
% \end{equation}

% Enable cancel symbol in equations
% \usepackage{cancel}
% Command template:
% \cancel{XXX}

% Enable probability P symbol in equations
% \usepackage{dsfont}
% Command template:
% \mathds{P}



% Algorithm Settings
\usepackage{algorithmic}
% \usepackage{algpseudocodex}
\renewcommand{\algorithmicrequire}{\textbf{Input:}}
\renewcommand{\algorithmicensure}{\textbf{Output:}}



% Enumerate Settings
% \usepackage{enumitem}
% \begin{enumerate}[label={(\arabic*).}]
%     \item XXX
%     \item XXX
% \end{enumerate}


% Line Number Settings
\usepackage{lineno}
% \linenumbers  % Uncomment this line to turn on the line number settings



% Code Settings
% \usepackage{courier}
% \usepackage{minted}

% Basic packages for inline and code block printings
% \usepackage{color,soul}
% \definecolor{codegray}{gray}{0.9}  % Define a new color for code block background.
% \usepackage{upquote}  % Ensure the ' mark is displayed correctly.
% \usepackage{listings}  % Used for code printing.

% Define an inline code display
% \newcommand{\codeinline}[1]{\colorbox{codegray}{\lstinline[basicstyle=\ttfamily,breaklines=true]|#1|}}

% Define a text field that can store the code without line-breaking. Note: hyperref package is required.
% % Set up a counter for auto-labelling
% \newcounter{codeCounter}
% % Setup the text field command. Check the hyperref package documentation for more set up options
% \newcommand{\codeblock}[1]
% {   
%     \stepcounter{codeCounter}
%     \TextField[default=\detokenize{#1}, width=0.9\columnwidth, height=15pt, charsize=12pt,
%     ]{\thecodeCounter}
% }
% % suppress the label behind the text field:
% \def\LayoutTextField#1#2{% label, field
%   #2%
% }
% Command template:
% \codeblock{Your codes here}



% Subfile Settings
% \usepackage{subfiles}
% \providecommand{\topdir}{.}
% \addglobalbib{\topdir/References.bib}



% Attach File Settings
\usepackage{attachfile}
% \attachfile[icon=Paperclip]{Test.pdf}



% Additional Settings

% Define a checkbox:
% \newcommand{\checkbox}[1]{%
%   \ifnum#1=1
%     \makebox[0pt][l]{\raisebox{0.15ex}{\hspace{0.1em}$\checkmark$}}%
%   \fi
%   $\square$%
% }

